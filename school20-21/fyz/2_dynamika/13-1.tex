% Options for packages loaded elsewhere
\PassOptionsToPackage{unicode}{hyperref}
\PassOptionsToPackage{hyphens}{url}
%
\documentclass[
]{article}
\usepackage{amsmath,amssymb}
\usepackage{lmodern}
\usepackage{ifxetex,ifluatex}
\ifnum 0\ifxetex 1\fi\ifluatex 1\fi=0 % if pdftex
  \usepackage[T1]{fontenc}
  \usepackage[utf8]{inputenc}
  \usepackage{textcomp} % provide euro and other symbols
\else % if luatex or xetex
  \usepackage{unicode-math}
  \defaultfontfeatures{Scale=MatchLowercase}
  \defaultfontfeatures[\rmfamily]{Ligatures=TeX,Scale=1}
\fi
% Use upquote if available, for straight quotes in verbatim environments
\IfFileExists{upquote.sty}{\usepackage{upquote}}{}
\IfFileExists{microtype.sty}{% use microtype if available
  \usepackage[]{microtype}
  \UseMicrotypeSet[protrusion]{basicmath} % disable protrusion for tt fonts
}{}
\makeatletter
\@ifundefined{KOMAClassName}{% if non-KOMA class
  \IfFileExists{parskip.sty}{%
    \usepackage{parskip}
  }{% else
    \setlength{\parindent}{0pt}
    \setlength{\parskip}{6pt plus 2pt minus 1pt}}
}{% if KOMA class
  \KOMAoptions{parskip=half}}
\makeatother
\usepackage{xcolor}
\IfFileExists{xurl.sty}{\usepackage{xurl}}{} % add URL line breaks if available
\IfFileExists{bookmark.sty}{\usepackage{bookmark}}{\usepackage{hyperref}}
\hypersetup{
  pdftitle={Dynamika},
  pdfauthor={Kapko},
  hidelinks,
  pdfcreator={LaTeX via pandoc}}
\urlstyle{same} % disable monospaced font for URLs
\usepackage[margin=1in]{geometry}
\usepackage{graphicx}
\makeatletter
\def\maxwidth{\ifdim\Gin@nat@width>\linewidth\linewidth\else\Gin@nat@width\fi}
\def\maxheight{\ifdim\Gin@nat@height>\textheight\textheight\else\Gin@nat@height\fi}
\makeatother
% Scale images if necessary, so that they will not overflow the page
% margins by default, and it is still possible to overwrite the defaults
% using explicit options in \includegraphics[width, height, ...]{}
\setkeys{Gin}{width=\maxwidth,height=\maxheight,keepaspectratio}
% Set default figure placement to htbp
\makeatletter
\def\fps@figure{htbp}
\makeatother
\setlength{\emergencystretch}{3em} % prevent overfull lines
\providecommand{\tightlist}{%
  \setlength{\itemsep}{0pt}\setlength{\parskip}{0pt}}
\setcounter{secnumdepth}{-\maxdimen} % remove section numbering
\ifluatex
  \usepackage{selnolig}  % disable illegal ligatures
\fi

\title{Dynamika}
\author{Kapko}
\date{}

\begin{document}
\maketitle

\hypertarget{dynamika-kmitaveho-pohybu}{%
\section{Dynamika kmitaveho pohybu}\label{dynamika-kmitaveho-pohybu}}

\begin{itemize}
\tightlist
\item
  \textbf{Skuma silu, ktora sposobuje kmitanie}
\item
  \textbf{Skuma suvislost medzi periodou kmitania a vlastnostami
  oscilatora}
\item
  budeme pouzivat \textbf{pruzinovy oscilator}

  \begin{itemize}
  \tightlist
  \item
    \textbf{Parametre}:
  \item
    hmotnost zavazia na pruzine
  \item
    vlastnost pruziny -\textgreater{} tuhost pruziny -\textgreater{}
    prejavuje sa pri deformacii -\textgreater{} aka velka sila je treba
    na to aby sa pruzina predlzila o 1 meter
  \end{itemize}
\end{itemize}

\(k = \frac{F_p}{\delta l}\)\\

\hypertarget{graf}{%
\subsection{1. Graf}\label{graf}}

\hypertarget{rovnovazna-poloha}{%
\subsubsection{1-1 Rovnovazna poloha}\label{rovnovazna-poloha}}

\(F_p = F_g\)\\
\(k \delta l = m * g\)\\
\(F_{vys} = F_p - F_g = 0 N\)\\

\hypertarget{pohbyb}{%
\subsubsection{1-2 Pohbyb}\label{pohbyb}}

\(F_p = k (\delta l - y)\)\\
\(F_g = m * g\)\\
\(F = F_p + F_g = k(\delta l - y) - m * g =\) \$ = k \delta l - m * g -
k * g\$ \(F = - k * g\)

\begin{itemize}
\tightlist
\item
  sila -\textgreater{} sposobuje kmitanie
\item
  sila -\textgreater{} priamoumerna okamzitej vychylke
\item
  sila -\textgreater{} je vzdy opacna ako vychylka
\item
  sila -\textgreater{} vzdy smeruje do rovnovaznej polohy
\end{itemize}

\hypertarget{suvislosti}{%
\subsection{2. Suvislosti}\label{suvislosti}}

\(m, k\)\\
\(F = - k * g\)\\
\(F = m * a = - m \omega^2 y\)\\
\(k = m \omega^2\)\\
\(\omega^2 = \frac{k}{m}\)\\
\(w = \sqrt{\frac{k}{m}}\) ~ \(f = \frac{1}{2 \pi} \sqrt{\frac{k}{m}}\)
-\textgreater{} Vlatna frekvencia ~ \(T = 2 \pi \sqrt{\frac{m}{k}}\) ~

\end{document}
