% Options for packages loaded elsewhere
\PassOptionsToPackage{unicode}{hyperref}
\PassOptionsToPackage{hyphens}{url}
%
\documentclass[
]{article}
\usepackage{lmodern}
\usepackage{amsmath}
\usepackage{ifxetex,ifluatex}
\ifnum 0\ifxetex 1\fi\ifluatex 1\fi=0 % if pdftex
  \usepackage[T1]{fontenc}
  \usepackage[utf8]{inputenc}
  \usepackage{textcomp} % provide euro and other symbols
  \usepackage{amssymb}
\else % if luatex or xetex
  \usepackage{unicode-math}
  \defaultfontfeatures{Scale=MatchLowercase}
  \defaultfontfeatures[\rmfamily]{Ligatures=TeX,Scale=1}
\fi
% Use upquote if available, for straight quotes in verbatim environments
\IfFileExists{upquote.sty}{\usepackage{upquote}}{}
\IfFileExists{microtype.sty}{% use microtype if available
  \usepackage[]{microtype}
  \UseMicrotypeSet[protrusion]{basicmath} % disable protrusion for tt fonts
}{}
\makeatletter
\@ifundefined{KOMAClassName}{% if non-KOMA class
  \IfFileExists{parskip.sty}{%
    \usepackage{parskip}
  }{% else
    \setlength{\parindent}{0pt}
    \setlength{\parskip}{6pt plus 2pt minus 1pt}}
}{% if KOMA class
  \KOMAoptions{parskip=half}}
\makeatother
\usepackage{xcolor}
\IfFileExists{xurl.sty}{\usepackage{xurl}}{} % add URL line breaks if available
\IfFileExists{bookmark.sty}{\usepackage{bookmark}}{\usepackage{hyperref}}
\hypersetup{
  pdftitle={Ulohy},
  pdfauthor={Jan Kapko},
  hidelinks,
  pdfcreator={LaTeX via pandoc}}
\urlstyle{same} % disable monospaced font for URLs
\usepackage[margin=1in]{geometry}
\usepackage{graphicx}
\makeatletter
\def\maxwidth{\ifdim\Gin@nat@width>\linewidth\linewidth\else\Gin@nat@width\fi}
\def\maxheight{\ifdim\Gin@nat@height>\textheight\textheight\else\Gin@nat@height\fi}
\makeatother
% Scale images if necessary, so that they will not overflow the page
% margins by default, and it is still possible to overwrite the defaults
% using explicit options in \includegraphics[width, height, ...]{}
\setkeys{Gin}{width=\maxwidth,height=\maxheight,keepaspectratio}
% Set default figure placement to htbp
\makeatletter
\def\fps@figure{htbp}
\makeatother
\setlength{\emergencystretch}{3em} % prevent overfull lines
\providecommand{\tightlist}{%
  \setlength{\itemsep}{0pt}\setlength{\parskip}{0pt}}
\setcounter{secnumdepth}{-\maxdimen} % remove section numbering
\usepackage{siunitx}
\ifluatex
  \usepackage{selnolig}  % disable illegal ligatures
\fi

\title{Ulohy}
\author{Jan Kapko}
\date{}

\begin{document}
\maketitle

\begin{enumerate}
\def\labelenumi{\arabic{enumi}.}
\tightlist
\item
  Priklad
\end{enumerate}

\hypertarget{zapis}{%
\section{Zapis:}\label{zapis}}

\(B = 80m\si{\tesla}\)\\
\(I = 2.5\si{\ampere}\)

\hypertarget{uloha-a}{%
\subsubsection{Uloha a:}\label{uloha-a}}

\[F = 12m\si{\newton}\] \[l = \frac{F_m}{B * I * \sin\alpha}\]
\[l = 6\si{\cm}\]

\hypertarget{uloha-b}{%
\subsubsection{Uloha b:}\label{uloha-b}}

\[F = 3m\si{\newton}\] \[\sin\alpha = \frac{F_m}{B * I * l}\]
\[\sin\alpha = 25\] \[\alpha = \ang{14,33}\]

\begin{enumerate}
\def\labelenumi{\arabic{enumi}.}
\setcounter{enumi}{1}
\tightlist
\item
  Priklad
\end{enumerate}

\hypertarget{zapis-1}{%
\section{Zapis}\label{zapis-1}}

\(I_1 = 6\si{\ampere}\)\\
\(I_2 = 8\si{\ampere}\)\\
\(l = 0.85m\)\\
\(d = 0.12m\)\\

\hypertarget{uloha-a-1}{%
\subsubsection{Uloha a:}\label{uloha-a-1}}

\[B = \frac{\mu_{0}}{2\pi} * \frac{I_1}{d}\]
\[B = 3,33 * 10^-6\si{\tesla}\]

\hypertarget{uloha-b-1}{%
\subsubsection{Uloha b:}\label{uloha-b-1}}

\[F_m = \frac{\mu_0}{2\pi} * \frac{I_1 * I_2}{d} * l\]
\[F_m = 6,8 * 10^-5\si{\newton}\]

\begin{enumerate}
\def\labelenumi{\arabic{enumi}.}
\setcounter{enumi}{2}
\tightlist
\item
  Priklad
\end{enumerate}

\hypertarget{zapis-2}{%
\section{Zapis}\label{zapis-2}}

\(B = 1,2\si{\tesla}\)\\
\(r = 0.018m\)\\
\(Q = 1.6 * 10^-19\si{\coulomb}\)\\
\(m_p = 1.7 * 10^-27 kg\)

\hypertarget{vypocet}{%
\subsubsection{Vypocet:}\label{vypocet}}

\[r = \frac{mv}{BQ}\]\\
\[\frac{1}{v} = \frac{m}{BQr}\]\\
\[v = \frac{BQr}{m}\]\\
\[v = 2.03 * 10^6 m/s\]

\begin{enumerate}
\def\labelenumi{\arabic{enumi}.}
\setcounter{enumi}{3}
\tightlist
\item
  Poznamky
\end{enumerate}

\hypertarget{elektromotor-na-jednosmerny-prud}{%
\section{Elektromotor na jednosmerny
prud}\label{elektromotor-na-jednosmerny-prud}}

\begin{itemize}
\tightlist
\item
  premiena elektricku energiu na mechanicku energiu
\item
  silove posobenie magnetickeho pola na zavit s prudom
\end{itemize}

\hypertarget{zlozenie}{%
\subsection{Zlozenie:}\label{zlozenie}}

\begin{itemize}
\tightlist
\item
  stator

  \begin{itemize}
  \tightlist
  \item
    permanentny magnet
  \end{itemize}
\item
  rotor

  \begin{itemize}
  \tightlist
  \item
    otacajuca sa cast elektormotora
  \end{itemize}
\item
  komutator

  \begin{itemize}
  \tightlist
  \item
    rotacny prepinac, dostava do rotora prud
  \end{itemize}
\end{itemize}

\hypertarget{fazy}{%
\subsection{Fazy:}\label{fazy}}

\begin{itemize}
\item
  \begin{enumerate}
  \def\labelenumi{\arabic{enumi}.}
  \tightlist
  \item
    faza:
  \end{enumerate}

  \begin{itemize}
  \tightlist
  \item
    rotor sa zacne posobenim sily otacat
  \end{itemize}
\item
  \begin{enumerate}
  \def\labelenumi{\arabic{enumi}.}
  \setcounter{enumi}{1}
  \tightlist
  \item
    faza:
  \end{enumerate}

  \begin{itemize}
  \tightlist
  \item
    rotor sa otocil o 45 stupnov, smer tocenia je rovnaky
  \end{itemize}
\item
  \begin{enumerate}
  \def\labelenumi{\arabic{enumi}.}
  \setcounter{enumi}{2}
  \tightlist
  \item
    faza:
  \end{enumerate}

  \begin{itemize}
  \tightlist
  \item
    rotator otoceny o 90 stupnov
  \item
    neposobia nan sily -\textgreater{} \textbf{mrtva poloha}
  \end{itemize}
\item
  \begin{enumerate}
  \def\labelenumi{\arabic{enumi}.}
  \setcounter{enumi}{3}
  \tightlist
  \item
    faza:
  \end{enumerate}

  \begin{itemize}
  \tightlist
  \item
    otocene o dalsich 45 stupnov
  \item
    rotator sa prepoloval a opat posobi magneticka sila v rovnakom smere
  \end{itemize}
\item
  \begin{enumerate}
  \def\labelenumi{\arabic{enumi}.}
  \setcounter{enumi}{4}
  \tightlist
  \item
    faza:
  \end{enumerate}

  \begin{itemize}
  \tightlist
  \item
    opat v prvej faze, tocia sa stale v rovnakom smere
  \item
    rozdiel je, ze poly su rozdielne
  \end{itemize}
\end{itemize}

\end{document}
